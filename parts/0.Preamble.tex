\title{SVD-based travel-time tomography with null-space exploiting regularization}
%\title{Directional beamforming with direction of arrival steering and noise covariance matrix estimation}

\author{%
	Vitor Gelsleichter Probst Curtarelli$^{\orcidlink{0009-0009-3996-5452}}$, Anderson Wedderhoff Spengler$^{\orcidlink{0000-0001-5204-2040}}$, \IEEEmembership{Fellow,~IEEE}, and Stephan Paul$^{\orcidlink{0000-0001-8181-1048}}$, \IEEEmembership{Fellow,~IEEE}
	%
	\thanks{\ieee Manuscript Info. Corresponding author: Vitor P. Curtarelli.}%
	\thanks{V. G.P. Curtarelli is with the Department of Electrical and Electronic Engineering at UFSC - Universidade Federal de Santa Catarina, Florianópolis, SC, Brazil (email: \url{vitor.curtarelli@posgrad.ufsc.br}).}%
	\thanks{A. W. Spengler is with the Department of Electrical and Electronic Engineering at UFSC - Universidade Federal de Santa Catarina, Florianópolis, SC, Brazil (email: \url{anderson.spengler@ufsc.br}).}%
	\thanks{S. Paul is with the Department of Mechanic Engineering at UFSC - Universidade Federal de Santa Catarina, Florianópolis, SC, Brazil (email: \url{stephan.paul@ufsc.br}).}%
	%
}
%\author{\ieee{}IEEE Publication Technology,~\IEEEmembership{Staff,~IEEE,}
%	% <-this % stops a space
%	\thanks{\ieee{}This paper was produced by the IEEE Publication Technology Group. They are in Piscataway, NJ.}% <-this % stops a space
%	\thanks{\ieee{}Manuscript received April 19, 2021; revised August 16, 2021.}}

% The paper headers
\markboth{\ieee{}Journal of \LaTeX\ Class Files,~Vol.~14, No.~8, August~2021}%
{Shell \MakeLowercase{\textit{et al.}}: A Sample Article Using IEEEtran.cls for IEEE Journals}

\IEEEpubid{\ieee{}0000--0000/00\$00.00~\copyright~2021 IEEE}
% Remember, if you use this you must call \IEEEpubidadjcol in the second
% column for its text to clear the IEEEpubid mark.

\maketitle

\begin{abstract}	
	In this paper, we propose a new singular-value decomposition -based approach for the inverse map estimation in travel-time tomography, for sound speed field reconstruction in 2D environments. This novel procedure allows exploiting the forward map's null-space for the regularization step (instead of the traditional additive term), ensuring the achievement of a primary metric's minimum, while controlling its behavior via the regularizing process. A standard method for the literature technique is also given, and a brief comparison of both is presented. Through simulations emulating oceanic bodies, the results show that our method more precisely models the speed field while achieving smaller errors.
\end{abstract}

\begin{IEEEkeywords}
	Travel-time tomography, singular value decomposition, sound speed field estimation
\end{IEEEkeywords}