\section{Simulation and Results}
\label{sec:simulation_and_results}
\def\sympath{h2212}
\def\ymin{1490.0}
\def\ymax{1555.0}
\def\nrows{31}
\def\ncols{11}
\def\pyNx{11}
\def\pyNz{31}
\def\pyN{341}

To compare in practice the techniques developed throughout the text, simulated environments were created, in which these techniques were employed. Namely, we will compare the literature standard technique from \cref{eq:sec2:minim_solution_literature} with the proposed method as obtained in \cref{eq:sec3:minim_solution_proposed}. In this section, we present the environment and practical conditions assumed, detail the considerations taken for the exhibited techniques, as well as other underlying necessary parameters, methods and metrics; and finally, we bring forth the results, and their following discussions.

\subsection{Environment conditions}

We assume that the environment is discretized into 2D cells along the width ($\x$-axis) and depth ($\z$-axis), with $N = N_{\x} \cdot N_{\z}$. This enables an easier visualization and faster computational complexity, without oversimplifying the presented conditions. For the simulations, we take $N_{\x} = \pyNx$, $N_{\z} = \pyNz$, and $N = \pyN$.

We also consider the sound speed field to be modeled as
\begin{equation}
	c(x,z) = 1500 + \Delta c_{\x}(x) + \Delta c_{\z}(z)
\end{equation}
where $c(x,z)$ is presented in \cref{subfig:sec4:discretized_ssf_with_profiles:fig1}, and the variations along width and depth are as shown in \cref{subfig:sec4:discretized_ssf_with_profiles:fig2,subfig:sec4:discretized_ssf_with_profiles:fig3}. Here, $\Delta$ merely symbolizes a variation. \Cref{subfig:sec4:discretized_ssf_with_profiles:fig1} also shows the chosen positions for the sources (in blue with $\bigcirc$, along the floor) and receivers (in red with $\times$, along the surface). This device positioning aligns with the practical underlying motivation, where ocean bottom nodes (OBNs) \cite{obns} are proposed to be used both as receivers for sub-surface imaging, and sources for the above-ground oceanic environment modeling; the later being necessary for a faithful reconstruction of the subterranean structures \cite{?}.

\begin{figure}[!htbp]
	\centering
	%	\tiny
	\begin{minipage}[c]{0.95\linewidth}
		\centering
		\begin{subfigure}{\linewidth}
			\centering
			\begin{tikzpicture}
				\begin{heatmap}{\ncols}[colorbar to name={velocityHeatmap}][fill opacity=0.8]
					\addplot3[surf, mesh/cols={\ncols}, mesh/rows={\nrows}, opacity=0.8] table[x=x, y=y, z=val, col sep=comma] {figures/data/\sympath/observed.csv};
					\addplot[only marks, mark=x, white, line width=2.8pt, mark size=0.5em+0.8pt] table [x=x_pos, y=y_pos, col sep=comma] {figures/data/\sympath/pos_receivers.csv};
					\addplot[only marks, mark=o, white, line width=2.8pt, mark size=0.4em] table [x=x_pos, y=y_pos, col sep=comma] {figures/data/\sympath/pos_sources.csv};
					\addplot[only marks, mark=x, red!50!black, very thick, mark size=0.5em] table [x=x_pos, y=y_pos, col sep=comma] {figures/data/\sympath/pos_receivers.csv};
					\addplot[only marks, mark=o, blue!50!black, very thick, mark size=0.4em] table [x=x_pos, y=y_pos, col sep=comma] {figures/data/\sympath/pos_sources.csv};
				\end{heatmap}
			\end{tikzpicture}
			\caption{Discretized sound speed field with device positions. Sources in blue with $\bigcirc$, receivers in red with $\times$.}
			\label{subfig:sec4:discretized_ssf_with_profiles:fig1}
		\end{subfigure}
		
		\vspace*{0.5em}
		\ref*{velocityHeatmap}
	\end{minipage}
	\\[1.5em]
	\begin{minipage}[c]{0.47\linewidth}
		\begin{subfigure}{\linewidth}
			\begin{tikzpicture}
				\begin{lineplot}{Width ($\si{\kilo\meter}$)}{$\Delta c_{\x}$ ($\si{\meter/\second}$)}{0}{5000}[xtick={0, 1000, ..., 5000}, xticklabels={0, 1, ..., 5}]
					\addplot[samples=100, domain=0:5000] {10*(2*x/5000-1)};
				\end{lineplot}
			\end{tikzpicture}
			\vspace*{-0.5em}
			\caption{Speed variation along the width.}
			\label{subfig:sec4:discretized_ssf_with_profiles:fig2}
		\end{subfigure}
	\end{minipage} \hfill
	\begin{minipage}[c]{0.47\linewidth}
		\begin{subfigure}{\linewidth}
			\begin{tikzpicture}
				\begin{lineplot}{Depth ($\si{\kilo\meter}$)}{$\Delta c_{\z}$ ($\si{\meter/\second}$)}{0}{5000}[xtick={0, 1000, ..., 5000}, xticklabels={0, 1, ..., 5}]
					\addplot[samples=100, domain=0:5000] {100*(1 - (exp((0.748*(5000-x)/5000)^2) - 0.748*(5000-x)/5000 - 0.773)/0.225)};
				\end{lineplot}
			\end{tikzpicture}
			\vspace*{-0.5em}
			\caption{Speed variation along the depth.}
			\label{subfig:sec4:discretized_ssf_with_profiles:fig3}
		\end{subfigure}
	\end{minipage}
	\caption{Discretized SSF, with horizontal and vertical profiles for speed variations.}
	\label{fig:sec4:discretized_ssf_with_profiles}
%	
%	\bigskip
%	\begin{subfigure}{0.95\linewidth}
%		\centering
%		\begin{tikzpicture}
%			\begin{heatmap}{\ncols}[colorbar to name={velocityHeatmapOpacity}][fill opacity=0.2]
%				\addplot3[surf, mesh/cols={\ncols}, mesh/rows={\nrows}, opacity=0.2] table[x=x, y=y, z=val, col sep=comma] {figures/data/h2835/observed.csv};
%				\addplot[only marks, mark=x, white, line width=2.4pt, mark size=0.5em+0.6pt] table [x=x_pos, y=y_pos, col sep=comma] {figures/data/\sympath/pos_receivers.csv};
%				\addplot[only marks, mark=o, white, line width=2.4pt, mark size=0.4em] table [x=x_pos, y=y_pos, col sep=comma] {figures/data/\sympath/pos_sources.csv};
%				\addplot[only marks, mark=x, red!50!black, very thick, mark size=0.5em] table [x=x_pos, y=y_pos, col sep=comma] {figures/data/\sympath/pos_receivers.csv};
%				\addplot[only marks, mark=o, blue!50!black, very thick, mark size=0.4em] table [x=x_pos, y=y_pos, col sep=comma] {figures/data/\sympath/pos_sources.csv};
%			\end{heatmap}
%		\end{tikzpicture}
%		
%		\vspace*{.5em}
%		\ref*{velocityHeatmapOpacity}
%	\end{subfigure}
%	\caption{Device positioning within the environment. Sources in blue with $\bigcirc$, receivers in red with $\times$.}
%	\label{fig:sec4:device_positioning}
\end{figure}

%\begin{figure}[!ht]
	\centering
	\begin{subfigure}{0.95\linewidth}
		\centering
		\begin{tikzpicture}
			\begin{heatmap}{\ncols}[colorbar to name={velocityHeatmapOpacity},width=0.9\linewidth,height=0.9\linewidth][fill opacity=0.2]
				\addplot3[surf, mesh/cols={\ncols}, mesh/rows={\nrows}, opacity=0.2] table[x=x, y=y, z=val, col sep=comma] {figures/data/h2835/observed.csv};
				\addplot[only marks, thick, mark=x, red!50!black, very thick, mark size=0.5em] table [x=x_pos, y=y_pos, col sep=comma] {figures/data/h2835/pos_receivers.csv};
				\addplot[only marks, thick, mark=o, blue!50!black, very thick, mark size=0.4em] table [x=x_pos, y=y_pos, col sep=comma] {figures/data/h2835/pos_sources.csv};
			\end{heatmap}
		\end{tikzpicture}
		
		\vspace*{.5em}
		\ref*{velocityHeatmapOpacity}
	\end{subfigure}
	\caption{Device positioning within the environment. Sources in blue with $\bigcirc$, receivers in red with $\times$.}
	\label{fig:sec4:device_positioning}
\end{figure}

Although the model variable $\bvs$ represents the environment's slowness, the figures and discussions will be with respect to the sound speed. Given their inverse relationship, this choice doesn't implicate in any information loss, only helping the visualization and presentability.

Given the amount of rays (\#receivers $\times$ \#source) and cells, this scenario is under the spatial over-modeling condition, as in \cref{subsec:spatial_under-over-_modeling}. Therefore, all simulations and analysis has to be done under the optics of its considerations.

The \texttt{EikonalFM} Python package \cite{ganster_eikonalfm_2025} was used, which implements the FMM \cite{sethian_fast_1999,chopp_improvements_2001} for solving the first arrival-times along the grid points. This aligns with the standard approach taken by the literature for estimating $\bvJ{i}$ at each iteration, given the slowness field $\bvs{i}$.

\subsection{Regularization matrix}

As discussed, a choice of regularization matrix will control which metric will be adjointed to the primary objective of SSF estimation. We chose a discrete Laplacian approach, improving smoothness in the outcome. We assume that $\bvs$ is constructed by vectorizing a $\bvS \in \re^{\sz{N_{\z}}{N_{\x}}}$ matrix, where each entry is its corresponding cell's slowness. More precisely, the relationship between elements of $\bvs$ and of $\bvS$ is
\begin{equation}
	\el{\bv{S}}[j,k] = \el{\bvs}[N_{\x}j+k],
\end{equation}
with $0 \leq j < N_{\z}$, and $0 \leq k < N_{\x}$. With this, $\bvD$ for this discrete Laplacian regularization can be written in block form
\begin{equation}
	\bvD = \begin{bmatrix}
		\bvQ' & \Id{N_{\x}} & \bv0  & \cdots & \bv0\\
		\Id{N_{\x}} & \bvQ & \Id{N_{\x}} & \cdots & \bv0\\
		\bv0 & \Id{N_{\x}} & \bvQ & \cdots & \bv0\\
		\vdots & \vdots & \vdots & \ddots & \vdots \\
		\bv0 & \bv0 & \bv0 & \cdots & \bvQ'
	\end{bmatrix}
\end{equation}
where $\bvQ,\bvQ' \in \re^{\sz{N_{\x}}{N_{\x}}}$ are block matrices, given by
\begin{subequations}
	\begin{gather}
		\bvQ' = \begin{bmatrix}
			-2 & 1 & 0 & \cdots & 0 & 0 \\
			1 & -3 & 1 & \cdots & 0 & 0 \\
			0 & 1 & -3 & \cdots & 0 & 0 \\
			\vdots & \vdots & \vdots & \ddots & \vdots & \vdots \\
			0 & 0 & 0 & \cdots & -3 & 1 \\
			0 & 0 & 0 & \cdots & 1  & -2
		\end{bmatrix}
	\end{gather}
	\begin{gather}
		\bvQ = \bvQ' - \Id{N_{\x}}
	\end{gather}
\end{subequations}

Alternatively, $\bvD$ is defined as having ones on hits first super-, first sub-, $N_{\x}$-th super, and $N_{\x}$-th sub-diagonals; and its main diagonal is such that any row (or column) sum is $0$. This ensures that cells on the boundary -- cells with 3 or less neighbors -- aren't over-represented in the discrete Laplacian.

\subsection{Metrics and evaluation}

The first metric reflects the primary optimization function, this being defined by $E(\bvs{i})$ as in \cref{eq:sec2:def_optimization_and_cost-function}. We will also use the root mean squared (RMS) error between the true velocity field $\bvc{\true}$ and the estimated one $\bvc = \nfrac{1}{\bvs}$,
\begin{equation}
	\mse(\bvs) = \frac{1}{N} \norm{\bvc{\true} - \frac{1}{\bvs}}
\end{equation}

$E(\bvs)$ reflects the model estimation of the cost function used, while $\mse(\bvs)$ represents the underlying objective of estimating the original SSF, as presented in \cref{subfig:sec4:discretized_ssf_with_profiles:fig1}. %One last metric of interest is the mean of the Laplacian squared, this being a measure of the roughness \cite{phillips_traveltime_1991} of the estimated field.

\subsection{Results}

\subsection{Discussion}