\subsection{Proofs and Theorems}
\label{app:appB:proofs}
We let $E(\bvs) = \norm{\bvR{i} \bvs - \bvt{\obs}}^2$, $\bvR{i} = \bvU \bvSi \bvV[T]$ be the SVD of $\bvR{i}$, and $\pinv*{\cdot}$ denote the Moore-Penrose pseudo-inverse. We also remark that $\rank{\bvR{i}} = R$.

\begin{theorem}{The p-inverse solution has zero gradient.}[\label{thm:p-inverse_zero-gradient}]
	Given a solution $\bvs = \pinv{\bvR{i}} \bvt{\obs}$, then the gradient of $E(\bvs)$ for this solution will be
	\begin{equation}
		\begin{split}
			\nabla E(\bvs{i+1})
			& = 2\pts{\bvR[T]{i} \bvR{i} \pinv{\bvR{i}} \bvt{\obs} - \bvR[T]{i} \bvt{\obs}} \\
			& = 2\pts{\bvV \bvSi[T] \bvU[T] \bvU \bvSi \bvV[T] \bvV \pinv{\bvSi} \bvU[T] \bvt{\obs} - \bvV \bvSi[T] \bvU[T] \bvt{\obs}} \\
			& = 2\pts{\bvV \bvSi[T] \bvSi \pinv{\bvSi} \bvU[T] \bvt{\obs} - \bvV \bvSi[T] \bvU[T] \bvt{\obs}} \\
			& = 2\pts{\bvV \bvSi[T] \bts{\Id{M;R} - \Id{M}} \bvU[T] \bvt{\obs}},
		\end{split}
	\end{equation}
	where $\Id{M;R} \in \re^{\sz{M}{M}}$ is a diagonal matrix with the first $R$ entries being $1$, and the remaining $M-R$ ones being $0$; and $\Id{M} \in \re^{\sz{M}{M}}$ is the identity matrix.
	
	From this, $\Id{M;R} - \Id{M}$ has its first $R$ diagonal entries $0$, and the last ones $-1$. Since all rows of $\bvSi[T]$ are in the null space of $\Id{M;R} - \Id{M}$ (only the first $R$ entries of any row of $\bvSi[T]$ are non-zero, exactly where $\Id{M;R} - \Id{M}$ is zero), then $\bvSi[T]\bts{\Id{M;R} - \Id{M}} = \bv0$. Therefore,
	\begin{equation}
		\begin{split}
			\nabla E(\pinv{\bvR{i}} \bvt{\obs})
			& = 2\pts{\bvV \,\bv0\, \bvU[T] \bvt{\obs}} \\
			& = \bv0.
		\end{split}
	\end{equation}
\end{theorem}

\begin{theorem}{The SVD truncated solution doesn't have zero gradient.}[\label{thm:svd-truncated_non-zero-gradient}]
	Given $\bvSi[t]$ be a truncated $\bvSi$ with only $\tilde{R}$ non-zero diagonal entries, then a solution to the minimization of $E(\bvs)$ of the form $\bvs = \bvV \pinv{\bvSi[t]} \bvU[T] \bvt{\obs}$ will have a gradient
	\begin{equation}
		\begin{split}
			\nabla E(\bvs{i+1})
			& = 2\pts{\bvV \bvSi[T] \bvU[T] \bvU \bvSi \bvV[T] \bvV \pinv{\bvSi[t]} \bvU[T] - \bvV \bvSi[T] \bvU[T]} \bvt{\obs} \\
			& = 2\pts{\bvV \bvSi[T] \bvSi \pinv{\bvSi[t]} \bvU[T] - \bvV \bvSi[T] \bvU[T]}\bvt{\obs} \\
			& = 2\bvV \bvSi[T] \pts{\Id{M;\tilde{R}} - \Id{M}} \bvU[T] \bvt{\obs} \\
			& = 2\pts{\sum_{i=\tilde{R}+1}^{R} \sigma_i \bvv{i} \bvu[T]{i}} \bvt{\obs},
		\end{split}
	\end{equation}
	which isn't identically $\bv0$.
\end{theorem}

\begin{theorem}{Difference in gradient for null-space exploiting truncated scenario.}[\label{thm:diff_gradient_null-space-exploiting_truncated}]
	We let $\bvV[t]{2}$ be the last $\tilde{R}$ columns of $\bvV$. With this,
	\begin{align*}
		& \nabla E(\bvG{i} \bvt{\obs} + \bvV{2} \bvmu) \\
		& = 2\bvR[T] \pts{\bvR \bvG{i} \bvt{\obs} + 2\bvR \bvV[t]{2} \bvmu - \bvt{\obs}} \\
		& = \nabla E(\bvG{i} \bvt{\obs}) + 2\bvR[T] \bvR \bvV[t]{2} \bvmu \\
		& = \nabla E(\bvG{i} \bvt{\obs}) + 2\bvV \bvSi[T] \bvU[T] \bvU \bvSi \bvV[T] \bvV[t]{2} \bvmu \\
		& = \nabla E(\bvG{i} \bvt{\obs}) + 2\bvV \bvSi[T] \bvSi \bvV[T] \bvV[t]{2} \bvmu \\
		& = \nabla E(\bvG{i} \bvt{\obs}) + 2\pts{\sum_{j=\tilde{R}+1}^{R} \lambda_j^2 \mu_{j-\tilde{R}} \bvv{j}}.
\end{align*}
\end{theorem}