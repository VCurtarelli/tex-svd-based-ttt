\section{Introduction}

Although the speed of sound $c$ in water is considered constant at around $1500 \si{\meter}/\si{\second}$, it can vary when considering environments of both small and large scales, respectively in small and large proportions \cite{kieffer_sound_1977,trusler_determination_2017,li_sound_2015,pinson_sound_2010}. Many properties can affect the speed of sound in water, such as pressure, temperature, salinity, and water currents \cite{wilson_speed_1959,medwin_fundamentals_1998,mcdougall_getting_2011,delgrosso_new_1974}. As these properties vary over the body of water, so does the velocity, creating a sound speed field (SSF) in the body of water. Although the term sound speed profile (SSP) is also commonly used \cite{li_acoustic_2019,li_sound_2015,pinson_sound_2010}, it more fittingly describes the 1D estimation scenario.

Initial works in the topic of estimating the velocity field in water aimed at estimating these key properties, as well as the relationship between them and the velocity of water. Works such as \cite{wilson_speed_1959,medwin_fundamentals_1998,mcdougall_getting_2011,delgrosso_new_1974,chen_speed_1977,wong_speed_1995,roquet_accurate_2015} operated under this framework, with some still being used today to model the behavior of water currents on a global scale.
The pursuit of knowing the speed of sound at each point of a body of water -- the SSF -- permits estimating the properties of a wave that travels through this medium  \cite{?}. By Snell's law \cite{snells-law} and Fermat's principle of least time \cite{fermats-least-time}, a change in speed of a wave also changes its direction. Therefore a wave that travels through a sufficiently large body of water will be also affected by it, resulting in curves, refractions, reflections, and formation/dispersion of beams. With this, we can estimate the path a unidirectional wave would course. By treating an omnidirectional wave as a cluster of unidirectional waves, we can calculate the travel-time between the source and any point within the medium, as well as its wavefront's profile. The knowledge of the path taken by an ensemble of rays is if great importance in the fields of ... and ... \cite{?,?}, and the estimation of the SSF within the environment is relevant for ..., ..., and ... \cite{?,?,?}

While traditional techniques used the environment properties -- pressure, temperature, and salinity -- to estimate the SSF, most modern SSF estimation approaches avoid estimating these properties, and treat this SSF estimation as an inversion problem. That is, since the knowledge of the SSF allows calculating the path of a wave through the medium (this being denominated the direct problem), then the knowledge of these paths admits the estimation of the SSF in the medium. Different techniques tackle this inversion problem by using measurements, and employing the acquired information through different lenses. Such techniques encompass Travel-Time Tomography \cite{bormann_seismic_2012,stefanov_travel_2019}, Matched Field Processing \cite{gemba_adaptive_2017,brienzo_broadband_1993} and Inversion \cite{dosso_bayesian_2011}, and Full Waveform Inversion \cite{virieux_overview_2009,fichtner_multiscale_2013}.

Among these, the travel-time tomography (TTT) is known for its simplicity and versatility \cite{aki_determination_1977,dines_computerized_1979,phillips_traveltime_1991}. It employs the difference in arrival time of the acoustic signals at the sensors, requiring a spatial distribution of sources and receivers for sufficient spatial resolution. In general, the SSF is discretized in layers and/or cells, where the sound speed is considered constant within each. The SSF's inversion iterates alternately between calculating the paths and lengths in each cell via solving the Eikonal equation \cite{fast-marching,fast-sweeping,ray-tracing}, and estimating the speed in each cell. Its simpleness, robustness, and computational efficiency are its main advantages, the later enabling real-time implementation. However, it depends on a precise positioning and syncing of the devices, and its discretization assumption results in limited resolution.

Many different approaches have been proposed for the TTT inversion, such as regularization \cite{aki_determination_1976,aki_determination_1977}, blurring \cite{phillips_traveltime_1991,ali_opensource_2019}, ray bending \cite{jeong_investigating_2024,hormati_robust_2010}, conjugate gradient minimization \cite{zhang_nonlinear_1998,tang_travel_2024}, and others \cite{zhang_nonlinear_1998,jeong_investigating_2024}. Each of these aim to enhance the inverse problem by correcting different issues that may appear, like ill-posed problems, physics-defiant models, or practically incoherent solutions.

In this paper, we propose a new approach, employing the singular-value decomposition (SVD) and the Moore-Penrose pseudo-inverse (from now on called p-inverse) for the primary inverse problem, as well as a null-space exploiting step for achieving a minimum on a secondary metric. These three techniques can be employed jointly to achieve an inverse mapping function that minimizes a primary cost function, and a secondary cost function within the null-space of the primary. This procedure ensures an arbitrarily small primary cost function (within the limits of the forward map), and a minimum on the secondary metric that doesn't have an impact on the primary.

Through simulations emulating real scenarios, the practical results show that the proposed framework more precisely models the observed SSF, as well as achieving smaller error values between observed and estimated travel times, and .... A computational complexity analysis shows that ....

This paper is organized as follows: in \cref{sec:ssf_estimation} we present the inverse problem's mathematical model as well as its physical underlying interpretation, and we establish the standard literature technique to which ours will be compared. In \cref{sec:svd_pseudo_ttt} we present the techniques to be used and develop the new proposed model, bringing a discussion on its theoretical advantages and disadvantages. In \cref{sec:simulation_and_results} we present the simulation conditions for comparing the presented techniques, as well as the results obtained through these simulations. \cref{sec:conclusion} concludes the paper.

Although the acronym SSF means sound speed field, mathematically it is more practical to model the problem via the slowness, the inverse of speed. Since both the sound speed field and sound slowness field would result in the same acronym, and both are simply inversely proportional, the SSF acronym will be used indistinguishably for both.
