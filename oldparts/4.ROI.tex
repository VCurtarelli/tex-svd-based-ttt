%\section{Region-of-Interest beamforming}
%As stated before, it is interesting to use the acquired information regarding the undesired source's direction of arrival to build a better beamformer. However, this direction of arrival is being estimated through the NCM estimation process, and is subject to imprecision. Thus, using an LCMV beamformer could lead to a wrong directional cancellation.
%
%We propose the use of a RONI beamformer to better filter out the undesired source's signal. We also assume some imprecision on the desired signal's DoA \textit{a priori} information, so a ROI beamformer will be used for it as well.
%
%\subsection{Regional constraints}
%Given a beamformer $\bvh$, the desired signal distortion index for a source with direction $\Theta = (\theta,\phi)$ is
%\begin{equations}
%	\di(\Theta)
%	& = \abs{\he{\bvd}(\Theta) \bvh - 1}^2 \\
%	& = \he{\pts{\he{\bvd}(\Theta) \bvh - 1}}
%	\pts{\he{\bvd}(\Theta) \bvh - 1}
%\end{equations}
%and, for a given region $\Omega$, the average distortion is
%\begin{equations}{eq:sec1:avg_distortion_Omega}
%	\bar{\di}(\Omega)
%	& = \frac{1}{\abs{\Omega}} \iint\limits_{\Omega} \di(\Theta) \dd\Theta \\
%	& = \frac{1}{\abs{\Omega}} \iint\limits_{\Omega} \he{\pts{\he{\bvd}(\theta,\phi) \bvh - 1}} \pts{\he{\bvd}(\theta,\phi) \bvh - 1} \cos\phi \dd\theta \dd\phi
%\end{equations}
%where we are using the double integral for spherical coordinates under the azimuth-elevation system, and not azimuth-polar system; with $\abs{\Omega}$ being
%\begin{equation}
%	\abs{\Omega} = \iint\limits_{\Omega} \cos\phi \dd\theta \dd\phi
%\end{equation}
%
%Manipulating \cref{eq:sec1:avg_distortion_Omega}, we get
%\begin{equation}
%	\label{eq:sec1:avg_distortion_Omega_simp}
%	\bar{\di}(\Omega) = \he{\bvh}\bvD[u]{\Omega}\bvh - \he{\bvh} \bvd[u]{\Omega} - \he{\bvd[u]{\Omega}} \bvh + 1
%\end{equation}
%in which
%\begin{subgather}
%	\bvD[u]{\Omega} = \frac{1}{\abs{\Omega}} \iint\limits_{\Omega} \bvd(\theta,\phi) \he{\bvd}(\theta,\phi) \cos\phi \dd\theta \dd\phi \\
%	\bvd[u]{\Omega} = \frac{1}{\abs{\Omega}} \iint\limits_{\Omega} \bvd(\theta,\phi) \cos\phi \dd\theta \dd\phi
%\end{subgather}
%
%Since we consider $\bvd(\Theta)$ to be a steering vector relative to the reference sensor, then $\bvd[u]{\Omega}$ is identical to the first column of $\bvD[u]{\Omega}$, as the delay for the first sensor (the reference sensor) is trivially 1.
%
%To minimize the average distortion for the region $\Omega$ w.r.t. the beamformer $\bvh$, we take the gradient of \cref{eq:sec1:avg_distortion_Omega_simp} and set it to 0, leading us to
%\begin{equation}
%	\label{eq:sec1:minimum-distortion_roi_constraint}
%	\bvD[u]{\Omega} \bvh = \bvd[u]{\Omega}
%\end{equation}
%named the \textit{minimum-distortion ROI constraint}.
%
%Given a beamformer $\bvh$, its noise signal reduction factor for a given direction $\Theta$ is
%\begin{equation}
%	\srf(\Theta) = \frac{1}{\he{\bvh} \bvd(\Theta) \he{\bvd}(\Theta) \bvh}
%\end{equation}
%and, for a region $\Agemo$,
%\begin{equation}
%	\label{eq:sec1:nsrf_roni}
%	\bar{\srf}(\Agemo) = \frac{1}{\he{\bvh} \,\bvD[u]{\Agemo}\, \bvh}
%\end{equation}
%
%To maximize the noise reduction over the region of non-interest $\Agemo$, we minimize the numerator of \cref{eq:sec1:nsrf_roni}, which results in the \textit{maximum-reduction RONI constraint}
%\begin{equation}
%	\bvD[u]{\Agemo} \,\bvh = \bv{0}_{\sz{M}{1}}
%\end{equation}
%
%\subsection{Maximum-gain beamformer}
%
%As an example, let's take the maximum-gain beamformer from Yehav's paper. In it (adapted to a frequency domain), the optimization problem is given by
%\begin{equation}
%%	\label{eq:sec1:maximum_gain_beamformer}
%	\bvh^{\star} = \arg\max_{\bvh} \frac{\he{\bvh} \bvD[u]{\Omega} \bvh}{\he{\bvh} \Corr{\bveta} \bvh}~\text{s.t.}~\bvD[u]{\Omega} \bvh = \bvd[u]{\Omega}~\text{and}~\bvD[u]{\Agemo} \bvh = \bv{0}
%\end{equation}
%which has the form of a generalized Rayleigh quotient, and can be solved as such. The matrices $\bvD[u]{\Omega}$ and $\Corr{\bveta}$ can be jointly diagonalized by a matrix $\bvT$, such that
%\begin{subalign}
%	\he{\bvT} \bvD[u]{\Omega} \bvT & = \bvLa \\
%	\he{\bvT} \Corr{\bveta} \bvT & = \bvI{M}
%\end{subalign}
%with $\bvLa$ being the (diagonal) eigenvalues matrix of $\inv{\Corr{\bveta}} \bvD[u]{\Omega}$, with its diagonal values being in decreasing order. Assuming that $\bvD[u]{\Omega}$ is a matrix with rank $P < M$, we will have $M-P$ zero entries in $\bvLa$'s main diagonal. With this, we define
%\begin{equation}
%	\bvT = \tup{\bvT{1}, \bvT{2}}
%\end{equation}
%where $\bvT{1}$ is a $\sz{M}{P}$ matrix corresponding to the $P$ non-zero eigenvalues, and $\bvT{2}$ corresponding to the $M-P$ zero entries. Since $\bvT$ is full-rank, we can define a vector $\bvg$ such that $\bvg = \inv{\bvT} \bvh$, and therefore
%\begin{equation}
%	\bvh = \bvT{1} \bvg{1} + \bvT{2} \bvg{2}
%\end{equation}
%with $\bvg{1}$ containing the first $P$ elements of $\bvg$, and $\bvg{2}$ the remaining $M-P$. Given an equivalence between $\bvh$ and $\bvg$, optimizing any of them is strictly equivalent to optimizing the other. Using all this, the minimization problem becomes
%\begin{equation}
%	\label{eq:sec4:max-gain_beamformer_opt-problem_gLg}
%	\bvg^{\star} = \arg\max_{\bvg} \frac{\he{\bvg{1}} \bvLa{1} \bvg{1}}{\he{\bvg{1}} \bvg{1} + \he{\bvg{2}} \bvg{2}}~\text{s.t.}~\bvLa{1} \bvg{1} = \he{\bvT{1}} \bvd[u]{\Omega}~\text{and}~\bvD[u]{\Agemo} \bvT{2} \bvg{2} = -\bvD[u]{\Agemo} \bvT{1} \bvg{1}
%\end{equation}
%in which $\bvLa{1}$ is the $\sz{P}{P}$ diagonal matrix with the non-zero entries of $\bvLa$. Since it is diagonalizable, the first constraint is trivially solved as
%\begin{equation}
%	\bvg{1} = \inv{\bvLa{1}} \he{\bvT{1}} \bvd[u]{\Omega} 
%\end{equation}
%
%Using this solution on the second constraint, and left-multiplying it by $\he{\bvT{2}}$, yields
%\begin{equation}
%	\he{\bvT{2}} \bvD[u]{\Agemo} \bvT{2} \bvg{2} = -\he{\bvT{2}} \bvD[u]{\Agemo} \bvT{1} \inv{\bvLa{1}} \he{\bvT{1}} \bvd[u]{\Omega}
%\end{equation}
%
%Note that the right-side of the equation is a constant $\sz{(M-P)}{1}$ vector, and as such we define $\bvup \equiv -\he{\bvT{2}} \bvD[u]{\Agemo} \bvT{1} \inv{\bvLa{1}} \he{\bvT{1}} \bvd[u]{\Omega}$. Assuming that $\he{\bvT{2}} \bvD[u]{\Agemo} \bvT{2}$ may not be a full-rank matrix (with rank $P' \leq M-P$), we diagonalize it as
%\begin{equation}
%	\he{\bvQ} \he{\bvT{2}} \bvD[u]{\Agemo} \bvT{2} \bvQ = \bvSi
%\end{equation}
%where $\bvSi$ is a diagonal eigenvalues matrix, with its diagonal entries being in decreasing order, and $\bvQ$'s columns are the orthonormal eigenvectors of $\he{\bvT{2}} \bvD[u]{\Agemo} \bvT{2} \bvg{2}$. We split $\bvQ = \tup{\bvQ{1}, \bvQ{2}}$, where $\bvQ{1}$ are the eigenvectors corresponding to the non-zero eigenvalues. Also, we define $\bvf = \tup{\bvf{1},\bv{2}}$, such that
%\begin{equation}
%	\bvg{2} = \bvQ{1} \bvf{1} + \bvQ{2} \bvf{2}
%\end{equation}
%and then the minimization constraint becomes
%\begin{equation}
%	\bvSi{1} \bvf{1} = \he{\bvQ{1}} \bvup
%\end{equation}
%where $\bvSi{1}$ is the diagonal matrix with the non-zero entries of $\bvSi$. The solution to this problem is
%\begin{equation}
%	\bvf{1} = -\inv{\bvSi} \he{\bvQ{1}} \he{\bvT{2}} \bvD[u]{\Agemo} \bvT{1} \inv{\bvLa{1}} \he{\bvT{1}} \bvd[u]{\Omega}
%\end{equation}
%leaving us with a minimization over $\bvf{2}$. Going back to the array gain equation, it's easy to see that
%\begin{equation}
%	\frac{\he{\bvh} \bvD[u]{\Omega} \bvh}{\he{\bvh} \Corr{\bveta} \bvh} = \frac{\he{\bvg{1}} \bvLa{1} \bvg{1}}{\he{\bvg{1}} \bvg{1} + \he{\bvf{1}} \bvf{1} + \he{\bvf{2}} \bvf{2}}
%\end{equation}
%where $\bvg{1}$ and $\bvf{1}$ are known. Trivially, the array gain is maximized iff $\bvf{2} = \bv{0}$. Therefore, the optimum beamformer $\bvh^\star$ is
%\begin{equation}
%%	\label{eq:sec4:cmg_beamformer_output}
%	\bvh^{\star} = \bvT{1} \bvg{1} + \bvT{2} \bvQ{1} \bvf{1}
%\end{equation}
%which, by substituting the previous results, leads to
%\begin{equations}
%	\bvh^{\star}
%	& = \bvT{1} \inv{\bvLa{1}} \he{\bvT{1}} \bvd[u]{\Omega} - \bvT{2} \bvQ{1} \inv{\bvSi} \he{\bvQ{1}} \he{\bvT{2}} \bvD[u]{\Agemo} \bvT{1} \inv{\bvLa{1}} \he{\bvT{1}} \bvd[u]{\Omega} \\
%	& = \pts{\Id{M} - \bvT{2} \bvQ{1} \inv{\bvSi} \he{\bvQ{1}} \he{\bvT{2}} \bvD[u]{\Agemo}} \bvT{1} \inv{\bvLa{1}} \he{\bvT{1}} \bvd[u]{\Omega}
%\end{equations}
%
%The first term in \cref{eq:sec4:cmg_beamformer_output} is responsible for achieving the minimum-distortion ROI constraint, while the second term lies on the null-space of the first, and maximizes the RONI canceling, while minimizing the overall output noise.

%\subsection{Minimum-noise beamformer}
%
%Another option for such a beamformer would be one that minimizes the noise in the output, instead of maximizing the gain. For distortionless beamformers, these two are equivalent, but since here we aim for a minimum distortion (instead of no distortion), in ROI beamforming they're different.
%
%The minimum-noise beamformer would be achieved through
%\begin{equation}
%	\label{eq:sec4:min-var_beamformer_opt-problem}
%	\bvh^{\star} = \arg\max_{\bvh} \frac{1}{\he{\bvh} \Corr{\bveta} \bvh}~\text{s.t.}~\bvD[u]{\Omega} \bvh = \bvd[u]{\Omega}~\text{and}~\bvD[u]{\Agemo} \bvh = \bv{0}
%\end{equation}
%which would be strictly equivalent to
%\begin{equation}
%	\bvh^{\star} = \arg\min_{\bvh} \he{\bvh} \Corr{\bveta} \bvh~\text{s.t.}~\bvD[u]{\Omega} \bvh = \bvd[u]{\Omega}~\text{and}~\bvD[u]{\Agemo} \bvh = \bv{0}
%\end{equation}
%
%Through the same process used for the maximum gain beamformer, we see that we'd achieve the same result as before, since in the maximum gain, after calculating $\bvg{1}$ the minimization only appears in the denominator. That is, applying the same thought process done between \cref{eq:sec1:maximum_gain_beamformer} and \cref{eq:sec4:max-gain_beamformer_opt-problem_gLg} to \cref{eq:sec4:min-var_beamformer_opt-problem}, yields
%\begin{equation}
%	\label{eq:sec4:min-var_beamformer_opt-problem_gLg}
%	\bvg^{\star} = \arg\max_{\bvg} \frac{1}{\he{\bvg{1}} \bvg{1} + \he{\bvg{2}} \bvg{2}}~\text{s.t.}~\bvLa{1} \bvg{1} = \he{\bvT{1}} \bvd[u]{\Omega}~\text{and}~\bvD[u]{\Agemo} \bvT{2} \bvg{2} = -\bvD[u]{\Agemo} \bvT{1} \bvg{1}
%\end{equation}
%where now the minimization problem is only on $\bvg{2}$, since $\bvg{1}$ is set under the first constraint. $\bvg{1}$ was obtained in the same way as for the maximum-gain beamformer, and so will $\bvg{2}$ be. Thus, both beamformers are identical.