\section{Signal model}
\label{sec:signal_model}

We consider a generic sensor array $S$ comprised of $M$ omnidirectional sensors within an echoic environment, populated by desired and contaminating sources. We assume this environment to be stationary both statistically and spatially, for ease of notation.

In the time domain, the observed signal in the $m-th$ sensor is $y_m(t)$, and it is modeled as
\begin{equation}
	\label{eq:sec2:definition_observed-signal_time-domain}
	y_m(t) = s_m(t) + n_m(t) + \sum_{i=1}^{I} w_{m,i}(t) + v_m(t)
\end{equation}
where $s_m(t)$ is the reverberant desired signal on the $m$-th sensor, $n_m(t)$ is the main undesired signal, $w_{m,i}(t)$ are other contaminating sources, and $v_m(t)$ is uncorrelated Gaussian noise.

Transforming all signals into the time-frequency domain, and decomposing the reverberant desired and main undesired signals into a sum of planar waves \cite{moore_compact_2022}, we have
\begin{equation}
	\hat{y}_m[l,k] = \sum_{j_x=1}^{J_x} x_{m,j_x}[l,k] + \sum_{j_p=1}^{J_p} \upsilon_{m,j_p}[l,k] + \sum_{i=1}^{I} w_{m,i}[l,k] + v_m[l,k]
\end{equation}
where $\hat{y}_m[l,k]$ is the approximate model of $y_m[l,k]$, under our assumptions.

Taking that each of the decomposed signals $s_m(t)$ and $u_m(t)$ is dominated by a single plane-wave \cite{yilmaz_blind_2004} (with index $j_x = j_p = 1$, which will be omitted), and assuming this to be the direct-path wave between the source and the $m$-th sensor, we write the observed signal as
\begin{equations}
	\dot{y}_m[l,k]
	& = d_{m}[k] x_{1}[l,k] + b_{m}[k] p_{1}[l,k] + \sum_{j_x=2}^{J_x} x_{m,j_x}[l,k] \\
	& + \sum_{j_p=2}^{J_p} p_{m,j_p}[l,k] + \sum_{i=1}^{I} w_{m,i}[l,k] + v_m[l,k]
\end{equations}

We call $x_1[l,k]$ the desired signal, $p_1[l,k]$ the interfering signal, $v_m[l,k]$ is the white noise, and the remaining ones the contaminating signals.

$d_m[k]$ is the relative frequency response between the reference (assumed $m=1$) and the $m$-th sensor's desired signal, given by
\begin{equation}
	\label{eq:sec2:definition_steering-vector_dmk}
	d_m[k] = e^{-\j 2\pi \frac{k f_0}{K\cdot c} r_m \cos(\theta_d - \psi_m) \cos(\phi_d - \lambda_m)}
\end{equation}
where: $f_0$ is the sampling frequency, $K$ is the number of frequency bins, $c$ is the wave speed, $(r_m,~\psi_m,~\lambda_m)$ are the spherical coordinates of the $m$-th sensor (distance, azimuth, elevation) assuming the reference sensor as the origin, and $(\theta_d,~\phi_d)$ is the interfering signal's angular direction. $b_m[k]$ is defined similarly to \cref{eq:sec2:definition_steering-vector_dmk}, with $\Theta = (\theta_b,~\phi_b)$ as its DoA.

By assuming that the contaminating signals $w_{m,i}[l,k]$ + non-direct path components of $x_{m,j_x}[l,k]$ and $p_{m,j_p}[l,k]$ can be approximated as a diffuse isotropic signal $\gamma_m[l,k]$ (as explained in \cite{moore_compact_2022}), we have
\begin{equation}
	\hat{y}_m[l,k] = d_m[k] x_1[l,k] + b_m[k] p_1[l,k] + \gamma_m[l,k] + v_m[l,k]
\end{equation}

We define $\bvy[h][l,k] = \tr{\left[\,\hat{y}_1[l,k],~\cdots,~\hat{y}_M[l,k]\,\right]}$ as the vectorization of the modeled observed signals, and $\bvd[k]$, $\bvb[k]$, $\bvga{}[l,k]$ and $\bvv[l,k]$ are defined similarly. Our sensor-stacked observed signal can also be written as
\begin{equation}
	\bvy[h][l,k] = \bvd[k] x_1[l,k] + \bvb[k] p_1[l,k] + \bvga{}[l,k] + \bvv{}[l,k]
\end{equation}

\subsection{Covariance matrix estimation}
We write the observed signal's correlation matrix as $\Corr{\bvy}[k]$, and it is given by (omitting the time-frequency indices for clarity)
\begin{equation}
	\Corr{\bvy} = \expec{\bvy \he{\bvy}}
\end{equation}
where $\he{\{\cdot\}}$ represents the conjugate-transpose operation. We also define $\Corr*{\bvy}[k]$ as the covariance matrix of the modeled observed signal $\bvy[h][l,k]$,
\begin{equations}
	\Corr*{\bvy}
	& = \bvd \he{\bvd} \var{x_1} + \bvb \he{\bvb} \var{p_1} + \Corr{\bvga} \var{\gamma_1} + \Corr{\bvv} \var{v_1} + \epsilon\Id \\
	& = \Corr{\bvx} \var{x_1} + \Corr{\bvp} \var{p_1} + \Corr{\bvga} \var{\gamma_1} + \Corr{\bvv} \var{v_1} + \epsilon\Id
\end{equations}
where $\var{x_1}[k]$ is the variance of $x_1[l,k]$ and $\Corr{\bvx}[l] = (\bvd \he{\bvd})[l]$ is its pseudo-normalized correlation matrix; and similarly for $\var{p_1}[k]$, $\var{\gamma_1}[k]$ and $\var{v_1}[k]$, as well as
$\Corr{\bvp}[l,k] = (\bvb \he{\bvb})[l,k]$, $\Corr{\bvga}[l,k]$ and $\Corr{\bvv}[l,k]$. $\epsilon$ is a normalization factor, to ensure a minimal white noise presence on the modeled matrix. $\Corr{\bvp}[k]$ is implicitly a function of $\Theta$.

Now, some assumptions are taken:
\begin{enumerate}
	\item $\bvd{}[l,k]$ is known;
	\item $\bvv{}[l,k]$ is a spatially uncorrelated white noise, such that $\Corr{\bvv}[l,k] = \Id$;
	\item $\bvga{}[l,k]$ is a spherically isotropic noise, and thus $\Corr{\bvga}[l,k]$ is known \cite{epain_spherical_2016};
	\item The DoA $\Theta$, which dictates $\bvb{}[l,k]$, is constant across the frequency spectrum;
\end{enumerate}

Although we take all (except $\Corr{\bvp}[k]$) correlation matrices to be known, the variances for all signals are not. Therefore, we have 4K+2 unknown in our problem: for each frequency, the variances for all signals at the reference sensor, $\var{x_1}$, $\var{u_1}$, $\var{\gamma_1}$ and $\var{v_1}$; and the DoA $\Theta$ which dictates the steering vector for the undesired signal.

%It is interesting to take into consideration some properties of the modeled correlation matrices:
%\begin{itemize}
%	\item The diagonal entries of $\Corr{\bvx}$, $\Corr{\bvu}$, $\Corr{\bvga}$ and $\Corr{\bvv}$ are unitary
%	\item For all entries, $\Corr{\bvv} = \Id$ and $\Corr{\bvga}$ are purely real matrices
%\end{itemize}

Given the values for the variances, as well as $\Theta$, the noise covariance matrix $\Corr{\bveta}[k]$ can be estimated, this being given by
\begin{equation}
	\label{eq:def_NCM_Corr-bveta}
	\Corr*{\bveta} = \Corr{\bvp} \var{p_1} + \Corr{\bvga} \var{\gamma_1} + \Corr{\bvv} \var{v_1} + \epsilon \Id
\end{equation}

Note that $\Corr*{\bveta}$ is frequency-dependent, this being omitted in \cref{eq:def_NCM_Corr-bveta} for space.
